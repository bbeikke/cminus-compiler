\documentclass[12pt]{amsart}
\usepackage{fullpage}

\title{CIS*4650 Checkpoint 2 report}
\author{Michael Melanson}

\begin{document}
\maketitle

\section{Summary}
The goal of this checkpoint was to add type inference and checking to
the C-Minus parser written for Checkpoint~1. It should check for
common errors, such as undefined variables, mismatched types, bad
function parameters, and so on.


\section{Building and running}

\subsection{Dependencies}
As described in Section~\ref{implementation}, the enclosed compiler
written in Haskell. Since a Haskell compiler is not standard on most
operating systems, the \verb|ghc| (Glasgow Haskell Compiler) package
may need to be installed. In addition, the Parsec ``parser combinator
library'' is required.

On Ubuntu or Debian, this can be done by running:
\begin{verse}
\verb|sudo apt-get install ghc haskell-parsec|
\end{verse}

On Fedora, CentOS, or other RPM-based distributions, this can be done
by running\footnote{I do not have access to such a system, so this is
  an educated guess; my understanding is that Parsec is included with
  GHC.}:
\begin{verse}
\verb|sudo yum install ghc|
\end{verse}

Other operating systems will have similar installation procedures.

\subsection{Building}
Internally, the Haskell build system Cabal is used to automate
building and packaging of the code. However, as per the assigment
specification, a Makefile has been included; this Makefile simply
invokes Cabal. The Makefile includes the following rules:

\begin{description}
\item[all] The default rule; invokes \emph{configure} then
  \emph{build}. 
\item[configure] Checks dependencies, prepares the \verb|dist| directory.
\item[build] Compiles and links the executable files.
\item[clean] Removes all generated files.
\item[docs] Generates source documentation (may require the additional
  \verb|haddock| package to be installed)
\end{description}

To compile the program, run \verb|make|. This produces two targets:
\verb|cm| and \verb|cm-test|, which are placed in their respective
directories in \verb|dist/build|. The \verb|cm-test| target, when
executed, runs a set of smoke tests against the compiler code. The
purpose of these tests is to verify that the compiler is not seriously
broken. They check that the basic program structures are parsed
correctly, and are fairly complete although not exhaustive. The main
interface to the compiler, however, is through the \verb|cm|
target. Its usage is detailed below.

\subsection{Invocation}
The compiler binary will be put in \verb|dist/build/cm/cm|. It
requires one of three flags, followed by a file path:

\begin{description}
\item[-a] Show the parsed abstract syntax tree.
\item[-s] Show symbol table contents
\item[-c] Check for errors only; either prints syntax errors or ``No
  errors'' if there are none.
\end{description}

There are also long forms of these flags, as explained if instead the
\verb|cm| binary is run with the \verb|--help| flag. If given any
other combination of arguments, it will print usage information and
terminate.


\section{Implementation notes}
\label{implementation}

As with Checkpoint~1, the code for this checkpoint is written in
Haskell, using the Parsec library to generate parsing tables. This
checkpoint is considerably longer than the last, at approximately 820
lines total, compared to 300 for the first checkpoint.

The abstract syntax tree implemented as part of Checkpoint~1 has not
been significantly modified. It has, however, been augmented with
positioning information via the \verb|Positioned| type
class\footnote{In Haskell, a type class is somewhat like a mixture of
  Java's generics and interfaces in that it defines a set of common
  features for any data type that instantiate it, while also requiring
  those instances to define the implementation of the
  features.}. Adding this information as a type class rather than as a
member of each of the datatypes allowed me to avoid significantly
altering the abstract data type. 

Most of the changes consisted of replacing calls to the function
\verb|return| in many of the parser combinators in \verb|Parser.hs|
with calls to \verb|returnWithPosition|. This function adds
positioning information to any value of types for which the
\verb|Positioned| type class is instanced unto.

Two other type classes were also added, called \verb|Nameable| and
\verb|Typeable|. These define the polymorphic \verb|typeOf| and
\verb|nameOf| functions respectively which are used throughout the type
checking code.

\section{Limitations}
The lack of error recovery mentioned in the Checkpoint~1 report is
still true for this checkpoint. Although I attempted to add error
recovery by adding \verb|manyR| and \verb|many1R| operators to Parsec,
which would work like Parsec's built-in \verb|many| and \verb|many1|
operators\footnote{See documentation at {\texttt
    http://hackage.haskell.org/packages/archive/parsec/3.0.0/doc/html/Text-ParserCombinators-Parsec-Combinator.html}}
except that they would perform error recovery on failure.




\end{document}
