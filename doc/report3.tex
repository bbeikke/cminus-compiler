\documentclass[12pt]{amsart}
\usepackage{fullpage}

\title{CIS*4650 Checkpoint 3 report}
\author{Michael Melanson}

\begin{document}
\maketitle

\section{Summary}
The goal of this checkpoint was to add code generation to the C-Minus
parser and type checker written for Checkpoints 1 and 2. It should
check for the same errors as in Checkpoint 2, but when there are no
errors, produce assembly code for the TM machine which implements the
program.

\section{Building and running}

\subsection{Dependencies}
As described in Section~\ref{implementation}, the enclosed compiler
written in Haskell. Since a Haskell compiler is not standard on most
operating systems, the \verb|ghc| (Glasgow Haskell Compiler) package
may need to be installed. In addition, the Parsec ``parser combinator
library'' is required.

On Ubuntu or Debian, this can be done by running:
\begin{verse}
\verb|sudo apt-get install ghc haskell-parsec|
\end{verse}

On Fedora, CentOS, or other RPM-based distributions, this can be done
by running\footnote{I do not have access to such a system, so this is
  an educated guess; my understanding is that Parsec is included with
  GHC.}:
\begin{verse}
\verb|sudo yum install ghc|
\end{verse}

Other operating systems will have similar installation procedures.

\subsection{Building}
Internally, the Haskell build system Cabal is used to automate
building and packaging of the code. However, as per the assigment
specification, a Makefile has been included; this Makefile simply
invokes Cabal. The Makefile includes the following rules:

\begin{description}
\item[all] The default rule; invokes \emph{configure} then
  \emph{build}. 
\item[configure] Checks dependencies, prepares the \verb|dist| directory.
\item[build] Compiles and links the executable files.
\item[clean] Removes all generated files.
\item[docs] Generates source documentation (may require the additional
  \verb|haddock| package to be installed)
\end{description}

To compile the program, run \verb|make|. This produces two targets:
\verb|cm| and \verb|cm-test|, which are placed in their respective
directories in \verb|dist/build|. The \verb|cm-test| target, when
executed, runs a set of smoke tests against the compiler code. The
purpose of these tests is to verify that the compiler is not seriously
broken. They check that the basic program structures are parsed
correctly, and are fairly complete although not exhaustive. The main
interface to the compiler, however, is through the \verb|cm|
target. Its usage is detailed below.

\subsection{Invocation}
The compiler binary will be put in \verb|dist/build/cm/cm|. It
requires one of three flags, followed by a file path:

\begin{description}
\item[-a] Show the parsed abstract syntax tree.
\item[-s] Show symbol table contents
\item[-c] Compile the program, producing assembly output if there are
  no errors
\end{description}

There are also long forms of these flags, as explained if instead the
\verb|cm| binary is run with the \verb|--help| flag. If given any
other combination of arguments, it will print usage information and
terminate.


\section{Implementation notes}
\label{implementation}

Code generation is implemented using a recursive, nested block-based
method. Local variable scope, stack use and labels are isolated by
pushing and popping ``compilation blocks'' from the compiler state
monad. Register allocation is not isolated, although it probably
should be.

Within a compilation block, compilation produces an intermediate code
that is the same as TM code except that some instructions include
unresolved symbol and label references as memory arguments. When a
block is complete, it is ``finalized'', at which time its labels and
symbols are resolved to relative addresses and the resulting
instructions are appended to the parent's scope.

A compilation block can return a value (through a register, which is
allocated in the parent block at the end of the child block).

\section{Limitations}

\subsection{Error recovery}
The lack of error recovery mentioned in the Checkpoint 1 and 2 reports
is still true for this checkpoint. However, multiple type errors are
reported when possible. Code generation does not produce any further
error conditions.

\subsection{Reachability analysis}
Code reachability analysis is not implemented. This means that the
compiler will often generate unreachable code, or not detect cases
where execution may exit a function without a \texttt{return}
statement. To avoid invalid output, the programmer should ensure that
every function ends with a return statement on all code paths. Return
statements are \emph{not} automatically added at the end of function
bodies.

\subsection{Register allocation}
Registers are allocated in a simple algorithm I call ``penny tray''
allocation, by analogy to penny trays in convenience stores, as in:
\begin{quote}
  ``{\it Need a register? Take a register. Have a register? Leave a
  register}''
\end{quote}

That is, available registers exist in a pool, and may be claimed
from that pool for use, and must be returned when no longer used. 

Memory spilling is not implemented. This means that large, complicated
algebraic expressions, especially those involving array accesses, may
exhaust all available registers. Expression evaluation is also rather
register-hungry, despite releasing registers as soon as
possible. Unlike in example code, expression evaluation is done
completely within registers; stack temporaries are not used.

If the system's registers are exhausted, the compiler will complain
about the lack of registers and terminate.


\end{document}
